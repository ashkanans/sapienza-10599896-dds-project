

\documentclass{article}
\usepackage{graphicx}

\title{Dependable Distributed Systems}
\author{Ashkan Ansarifard}
\date{Start date: February 11, 2023 \\ Last modified: \today}

\begin{document}
	
	\maketitle
	\newpage
	\tableofcontents
	
	\newpage
	\newpage
	\section{Introduction}
	There are two main challenges in Dynamism:
	1. Define an allowable fialure while ensuring the liveness
	2. Ensuring safety in case of concurrent reconfiguration
	
	\newpage
	\section{Time in Distributed Systems}
	
	\newpage
	\section{Logical Clock}
	
	\newpage
	\section{Distributed Mutual Exclusion}
	
	\newpage
	\section{Failure Detection Abstraction}
	A \textbf{failure detector abstraction} is a software module used to detect faulty processes, it encapsulate timing assumptions of a either partially synchronous or fully synchronous system. It has two properties:
	\begin{itemize}
		\item \textbf{Accuracy}: that represents the ability to avoid mistakes
		\item \textbf{Completeness}: that represents the ability to detect all failures
	\end{itemize}
	
	\subsection{Perfect Failure Detectors}
	\label{sec:perfect-failure-detectors}
	Here is a figure that illustrates the specifications of perfect failure detectors:

	
	\subsection{Eventually Perfect Failure Detectors}
	
	\subsection{Leader Election}
	
	\subsection{Eventual Leader Election}
	
	\newpage
	\section{Broadcast Communications}
	
	\newpage
	\section{Consensus}
	
	\newpage
	\section{Ordered Communications}
	
	\newpage
	\section{Registers}
	
	\newpage
	\section{Software Replication}
	
	\newpage
	\section{Overview on Capacity Planning}
	
	\newpage
	\section{Modeling the Workload of a System}
	
	\newpage
	\section{Building a Performance Model 1}
	
	\newpage
	\section{Building a Performance Model 2}
	
	\newpage
	\section{Dependability Evaluation}
	
	\newpage
	\section{Intro to Experimental Design}
	
	\newpage
	\section{CAP Theorem}
	
	\newpage
	\section{Consistency Criteria for Distributed Shared Memories}
	
	\newpage
	\section{Publish-Subscribe Communication Paradigm}
	
	\newpage
	\section{Overlay Networks}
	
	\newpage
	\section{DLT and Blockchain}
	
	\newpage
	\section{Exercises}
	\emph{Notice: The exercises are from 2022-2023 academic year.}
	
	\newpage
	\section{Exams}
	
	
\end{document}\dfrac{num}{den}